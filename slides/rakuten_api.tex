% Created 2014-09-24 水 21:07
\documentclass[t]{beamer}
\usepackage{zxjatype}
\usepackage[ipa]{zxjafont}
\setbeamertemplate{navigation symbols}{}
\hypersetup{colorlinks,linkcolor=,urlcolor=gray}
\AtBeginSection[]
{
  \begin{frame}<beamer>{Outline}
  \tableofcontents[currentsection,currentsubsection]
  \end{frame}
}
\setbeamertemplate{navigation symbols}{}
\usepackage[utf8]{inputenc}
\usepackage[T1]{fontenc}
\usepackage{fixltx2e}
\usepackage{graphicx}
\usepackage{longtable}
\usepackage{float}
\usepackage{wrapfig}
\usepackage{rotating}
\usepackage[normalem]{ulem}
\usepackage{amsmath}
\usepackage{textcomp}
\usepackage{marvosym}
\usepackage{wasysym}
\usepackage{amssymb}
\usepackage{hyperref}
\tolerance=1000
\usepackage{minted}
\institute[AIIT]{産業技術大学院大学(AIIT)}
\usetheme{Berkeley}
\usecolortheme{whale}
\setcounter{secnumdepth}{2}
\author{中鉢 欣秀・上田 隆一}
\date{}
\title{楽天APIを利用したアプリケーション}
\hypersetup{
  pdfkeywords={},
  pdfsubject={},
  pdfcreator={Emacs 24.3.2 (Org mode 8.2.5h)}}
\begin{document}

\maketitle

\section{楽天API}
\label{sec-1}
\begin{frame}[label=sec-1-1]{楽天APIとは?}
\begin{itemize}
\item \href{http://webservice.rakuten.co.jp/document/}{楽天ウェブサービス: API一覧}
\end{itemize}
\end{frame}

\begin{frame}[fragile,label=sec-1-2]{アプリの入れ物を作る}
 \begin{itemize}
\item \texttt{rakuten\_enpit} リポジトリを作成
\item Herokuに接続
− アプリURLを取得
\end{itemize}

\begin{minted}[]{bash}
mkdir ~/rakuten_enpit
cd ~/rakuten_enpit
git init
git create
heroku create
\end{minted}
\end{frame}
\begin{frame}[label=sec-1-3]{アプリIDの発行}
\begin{itemize}
\item 新規アプリを登録する
\begin{itemize}
\item \href{https://webservice.rakuten.co.jp/app/create}{楽天ウェブサービス: 新規アプリ登録}
\end{itemize}
\item アプリ名(任意),アプリのURL,認証コードを入力
\begin{itemize}
\item アプリID,アフィリエイトID等を控えておく
\end{itemize}
\end{itemize}
\end{frame}
\begin{frame}[fragile,label=sec-1-4]{環境変数の設定}
 \begin{itemize}
\item アプリID(APPID)とアフィリエイトID(AFID)を環境変数に登録
\item \texttt{\textasciitilde{}/.bash\_profile} に次の行を追加(自分のID等に書き換えること)
\item \texttt{exit} して,再度 \texttt{vagrant ssh}
\end{itemize}

\begin{minted}[]{bash}
export APPID=102266705971259xxxx
export AFID=11b23d92.8f6b6ff4.11b23d93.???????
\end{minted}
\end{frame}
% Emacs 24.3.2 (Org mode 8.2.5h)
\end{document}