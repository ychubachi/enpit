% Created 2014-09-21 日 14:13
\documentclass[t]{beamer}
\usepackage{zxjatype}
\usepackage[ipa]{zxjafont}
\setbeamertemplate{navigation symbols}{}
\hypersetup{colorlinks,linkcolor=,urlcolor=gray}
\AtBeginSection[]
{
  \begin{frame}<beamer>{Outline}
  \tableofcontents[currentsection,currentsubsection]
  \end{frame}
}
\setbeamertemplate{navigation symbols}{}
\usepackage[utf8]{inputenc}
\usepackage[T1]{fontenc}
\usepackage{fixltx2e}
\usepackage{graphicx}
\usepackage{longtable}
\usepackage{float}
\usepackage{wrapfig}
\usepackage{rotating}
\usepackage[normalem]{ulem}
\usepackage{amsmath}
\usepackage{textcomp}
\usepackage{marvosym}
\usepackage{wasysym}
\usepackage{amssymb}
\usepackage{hyperref}
\tolerance=1000
\usepackage{minted}
\institute[AIIT]{産業技術大学院大学(AIIT)}
\usetheme{Berkeley}
\usecolortheme{whale}
\setcounter{secnumdepth}{2}
\author{中鉢 欣秀 \\ yc@aiit.ac.jp}
\date{}
\title{自己紹介}
\hypersetup{
  pdfkeywords={},
  pdfsubject={},
  pdfcreator={Emacs 24.3.2 (Org mode 8.2.5h)}}
\begin{document}

\maketitle

\section{自己紹介}
\label{sec-1}
\begin{frame}[label=sec-1-1]{自己紹介}
\begin{block}{名前}
\begin{itemize}
\item 中鉢 欣秀(ちゅうばち よしひで)
\end{itemize}
\end{block}

\begin{block}{出身地}
\begin{itemize}
\item 宮城県仙台市
\end{itemize}
\end{block}

\begin{block}{肩書}
\begin{itemize}
\item 産業技術大学院大学 産業技術研究科 \\ 情報アーキテクチャ専攻 准教授
\end{itemize}
\end{block}
\end{frame}

\begin{frame}[label=sec-1-2]{連絡先}
\begin{description}
\item[{E-Mail}] yc@aiit\ldots{}
\item[{Facebook}] ychubachi
\item[{Twitter}] ychubachi (あんまり使ってない)
\item[{Skype}] ychubachi (あんまり使ってない)
\end{description}
\end{frame}

\begin{frame}[label=sec-1-3]{学歴}
\begin{center}
\begin{tabular}{lll}
1991年 & 4月 & 慶應義塾大学環境情報学部 入学\\
1995年 & \alert{10月} & 同大大学院 政策・メディア研究科\\
 &  & 修士課程 入学\\
1997年 & 10月 & 同大大学院 政策・メディア研究科\\
 &  & 後期博士課程 入学\\
2004年 & 10月 & 同大大学院 政策・メディア研究科\\
 &  & 後期博士課程 卒業\\
 &  & 学位:博士(政策・メディア)\\
\end{tabular}
\end{center}
\end{frame}
\begin{frame}[label=sec-1-4]{職歴}
\begin{center}
\begin{tabular}{lll}
1997年 & 10月 & 合資会社ニューメリック設立\\
 &  & \alert{社長就任}\\
2005年 & 4月 & 独立行政法人科学技術振興機構\\
 &  & PD級研究員\\
 &  & (長岡技術科学大学)\\
2006年 & 4月 & 産業技術大学院大学 産業技術研究科\\
 &  & 情報アーキテクチャ専攻 准教授\\
\end{tabular}
\end{center}
\end{frame}

\begin{frame}[label=sec-1-5]{起業経験}
\begin{block}{社名}
\begin{itemize}
\item 合資会社ニューメリック
\end{itemize}
\end{block}

\begin{block}{設立}
\begin{itemize}
\item 1997年
\end{itemize}
\end{block}

\begin{block}{資本金}
\begin{itemize}
\item \alert{18万円}
\end{itemize}
\end{block}
\end{frame}

\begin{frame}[label=sec-1-6]{起業の背景}
\begin{block}{設立当時の状況}
\begin{itemize}
\item Windows 95が普及(初期状態でインターネットは使えなかった)
\item 後輩のやっていたベンチャーの仕事を手伝って面白かった
\end{itemize}
\end{block}

\begin{block}{会社設立の理由}
\begin{itemize}
\item 「やってみたかった」から
\item 少しプログラムがかければ仕事はいくらでもあった
\item 後輩にそそのかされた・笑
\end{itemize}
\end{block}
\end{frame}

\begin{frame}[label=sec-1-7]{起業から学んだこと}
\begin{itemize}
\item 実プロジェクトの経験
\item 使える技術
\item お金は簡単には儲からない
\end{itemize}
\end{frame}

\begin{frame}[label=sec-1-8]{教育における関心事}
\begin{block}{情報技術産業の変化}
\begin{itemize}
\item 情報技術のマーケットが変化
\item ユーザ・ベンダ型モデルの終焉
\end{itemize}
\end{block}

\begin{block}{モダンなソフトウエア開発者}
\begin{itemize}
\item 新しいサービスの企画から,ソフトウエアの実装まで何でもこなせる開発者
\item このような人材の育成方法
\end{itemize}
\end{block}
\end{frame}
% Emacs 24.3.2 (Org mode 8.2.5h)
\end{document}